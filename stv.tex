\documentclass[review]{elsarticle}

\usepackage{lineno,hyperref}
\usepackage{pdflscape}
\modulolinenumbers[5]

\journal{Geochemica and Cosmochemica Acta}

%%%%%%%%%%%%%%%%%%%%%%%
%% Elsevier bibliography styles
%%%%%%%%%%%%%%%%%%%%%%%
%% To change the style, put a % in front of the second line of the current style and
%% remove the % from the second line of the style you would like to use.
%%%%%%%%%%%%%%%%%%%%%%%

%% Numbered
%\bibliographystyle{model1-num-names}

%% Numbered without titles
%\bibliographystyle{model1a-num-names}

%% Harvard
\bibliographystyle{model2-names.bst}\biboptions{authoryear}

%% Vancouver numbered
%\usepackage{numcompress}\bibliographystyle{model3-num-names}

%% Vancouver name/year
%\usepackage{numcompress}\bibliographystyle{model4-names}\biboptions{authoryear}

%% APA style
%\bibliographystyle{model5-names}\biboptions{authoryear}

%% AMA style
%\usepackage{numcompress}\bibliographystyle{model6-num-names}

%% `Elsevier LaTeX' style
%\bibliographystyle{elsarticle-num}
%%%%%%%%%%%%%%%%%%%%%%%

\begin{document}

\begin{frontmatter}

\title{Quenchable water-rich aluminous post-stishovite, and implications for seismic scatterers in the lower mantle}
%\tnotetext[mytitlenote]{Fully documented templates are available in the elsarticle package on \href{http://www.ctan.org/tex-archive/macros/latex/contrib/elsarticle}{CTAN}.}

%% Group authors per affiliation:
\author{R. Myhill, D. J. Frost, T. Boffa-Ballaran, H. Bureau, C. Raepsaet}
\address{Bayerisches Geoinstitut, Universit\"{a}t Bayreuth, Universit\"{a}tsstrasse 30, 95447 Bayreuth, Germany}
\cortext[mycorrespondingauthor]{Corresponding author: R. Myhill}
\ead{myhill.bob@gmail.com}

\begin{abstract}

\end{abstract}

\begin{keyword}
high pressure \sep post-stishovite \sep water \sep slab \sep scatterers
\end{keyword}

\end{frontmatter}

\linenumbers

\section{Introduction}

The stishovite structure (tetragonal, P4$_2$/mnm) distorts slightly to the CaCl$_2$-type structure (orthorhombic, Pnnm) at 50 GPa. A few studies have suggested that the CaCl$_2$ type may undergo a phase transition to a mineral named seifertite with the scrutinyite ($\alpha$-PbO$_2$ )structure (orthorhombic, Pbcn or Pb2n) within mantle pressures

\cite{TY1989}
\cite{AFGH1998}
\cite{KCHM1995}
\cite{OHMI2002}
\cite{Lakshtanovetal2007}
\cite{Lakshtanovetal2005}
\cite{SAP2012}
\cite{PS2004}
\cite{BBB2006}
\cite{LKSOLBI2007}
\cite{CK2002}
\cite{HP2011}
\cite{SLB2011}
\cite{CHRU2014}

\section{Experimental and analytical techniques}

\section{Chemical composition}
\begin{figure}[ht!]
  \centering
  %\includegraphics[width=0.8\textwidth]{figures/}
  \caption{Single crystal XRD spectra of post-stishovite}
  \label{fig:xrd_post_stv}
\end{figure}

\section{Conclusions}



\clearpage
\section*{References}

\bibliography{references_stv}

\end{document}
